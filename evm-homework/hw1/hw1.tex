\documentclass{article}

\usepackage[utf8]{inputenc}
\usepackage[T2A]{fontenc}
\usepackage[russian,english]{babel}
\usepackage{fontspec}
\usepackage{amsfonts}

\usepackage{enumitem}
\usepackage{amsmath}
\usepackage{amsthm}
\usepackage{listings}

\usepackage[left=2cm,right=2cm,bottom=2cm,top=2cm]{geometry}

\setmainfont{PT Serif}

\title{Домашняя работа по архитектуре ЭВМ № 1. Вариант 7. Кривенко Андрей М3107}

\begin{document}
\maketitle
\[ № 1 \]
\[ A = 6582 \]
\[ C = 17436 \]
\begin{enumerate}
  \item $ X_{1} = A = 6582 \; \; B_{1} = 0001 \; 1001 \; 1011 \; 0110 $ \\ 
 $ 6582_{10} = 2 + 4 + 16 + 32 + 128 + 256 + 2048 + 4096 = 1 \; 1001 \; 1011 \; 0110_{2} $
 $ 1 \; 1001 \; 1011 \; 0110_{2} = 2^{1} + 2^{2} +2^{4} + 2^{5} + 2^{7} + 2^{8} + 2^{11} + 2^{12} = 6582_{10} $
  \item $ X_{2} = C = 17436 \; \; B_{2} = 0100 \; 0100 \; 0001 \; 1100 $
  \item $ X_{3} = A + C = 24018 \; \; B_{3} = 0101 \; 1101 \; 1101 \; 0010 $
  \item $ X_{4} = A + C + C = 41454 \; \; B_{4} = 1010 \; 0001 \; 1110 \; 1110 $ \\
  Происходит переполнение разрядной сетки
  \item $ X_{5} = C - A = 10854 \; \; B_{5} = 0010 \; 1010 \; 0110 \; 0110 $
  \item $ X_{6} = 65536 - X_{4} = 24082 \; \; B_{6} = 0101 \; 1110 \; 0001 \; 0010 $
  \item $ X_{7} = -X_{1} = -6582 \; \; B_{7} = 1110 \; 0110 \; 0100 \; 1010 $ \\
  $ 6582_{10} = 0001 \; 1001 \; 1011 \; 0110 $ \\
  Найдем дополнительный код: \\
  Инвертирем каждый бит числа: $ 1110 \; 0110 \; 0100 \; 1001_{2} $ \\
  Прибаваляем единицу: $ 1110 \; 0110 \; 0100 \; 1010_{2} $
  \item $ X_{8} = -X_{2} = -17436 \; \; B_{8} = 1011 \; 1011 \; 1110 \; 0100 $
  \item $ X_{9} = -X_{3} = -24018 \; \; B_{9} = 1010 \; 0010 \; 0010 \; 1110 $
  \item $ X_{10} = -X_{4} = -41454 \; \; B_{10} = 0101 \; 1110 \; 0001 \; 0010 $ \\
  Происходит переполнение разрядной сетки
  \item $ X_{11} = -X_{5} = -10854 \; \; B_{11} = 1101 \; 0101 \; 1001 \; 1010 $
  \item $ X_{12} = -X_{6} = -24082 \; \; B_{12} = 1010 \; 0001 \; 1110 \; 1110 $
\end{enumerate}
\[ № 2 \]
\begin{enumerate}
  \item $ B1 \; +  B2 = 0001 \; 1001 \; 1011 \; 0110 \; + \; 0100 \; 0100 \; 0001 \; 1100 = 0101 \; 1101 \; 1101 \; 0010 _{2} = 24018_{10}  $ \\
  $ X_{1} \; + \; X_{2} \; = 24018_{10} $
  \item $ B_{2} \; + \; B_{3} = 0100 \; 0100 \; 0001 \; 1100 \; + \; 0101 \; 1101 \; 1101 \; 0010 = 1010 \; 0001 \; 1110 \; 1110_{2} = -24082_{10}  $ \\
  $ X_{2} \; + \; X_{3} \; = 41454_{10} $ \\
  Происходит переполнение разрядной сетки
  \item $ B_{7} \; + \; B_{8} = 1110 \; 0110 \; 0100 \; 1010 \; + \; 1011 \; 1011 \; 1110 \; 0100 = 1010 \; 0010 \; 0010 \; 1110_{2} = -24018_{10}  $ \\
  $ X_{7} \; + \; X_{8} \; = -24018_{10} $
  \item $ B_{8} \; + \; B_{9} = 1011 \; 1011 \; 1110 \; 0100 \; + \; 1010 \; 0010 \; 0010 \; 1110 = 0101 \; 1110 \; 0001 \; 0010_{2} = 24082_{10} $ \\
  $ X_{8} \; + \; X_{9} \; = -41454_{10} $ \\
  Происходит переполнение разрядной сетки 
  \item $ B_{2} \; + \; B_{7} = 0100 \; 0100 \; 0001 \; 1100 \; + \; 1110 \; 0110 \; 0100 \; 1010 = 0010 \; 1010 \; 0110 \; 0110_{2} = 10854_{10} $ \\
  $ X_{2} \; + \; X_{7} \; = 10854_{10} $
  \item $ B_{1} \; + \; B_{8} = 0001 \; 1001 \; 1011 \; 0110 \; + \; 1011 \; 1011 \; 1110 \; 0100 = 1101 \; 0101 \; 1001 \; 1010_{2} = -10854_{10} $ \\
  $ X_{1} \; + \; X_{8} \; = -10854_{10} $
\end{enumerate}
\end{document}
