\documentclass{article}

\usepackage[utf8]{inputenc}
\usepackage[T2A]{fontenc}
\usepackage[russian,english]{babel}
\usepackage{fontspec}
\usepackage{amsfonts}

\usepackage{enumitem}
\usepackage{amsmath}
\usepackage{amsthm}
\usepackage{listings}

\setmainfont{PT Serif}

\title{ДЗ 5 по алгоритмам. Кривенко Андрей. М3107}

\begin{document}
\maketitle
\begin{enumerate}
  \item inorder traversal
    \begin{lstlisting}[language=c++]
    void printSorted(Node* x) {
      if (x != nullptr) {
        printSorted(x->left);
        printf("%d ", x->key);
        printSorted(x->right);
      }
    }
  \end{lstlisting} 
  \item Преобразую рекурсивное решение предыдущей задачи
    \begin{lstlisting}[language=c++]
    void printSorted(Node* x) {
      Node* tmp = x;
      while (true) {
        //спускаемся максимально влево
        while (tmp->left != nullptr) {
          tmp = tmp->left;
        }

        printf("%d ", tmp->key);
        
        //теперь нужно найти элемент справа
        if (tmp->right != nullptr) {
          tmp = tmp->right;  
        } else {
          while (tmp->right != nullptr) {
            tmp = tmp->parent;
          }
        }
      }
    }
  \end{lstlisting} 
  \item  postorder traversal - считаем сколько слева, затем считаем, сколько справа
  \begin{lstlisting}[language=c++]
    int w(Node* x) {
      if (x != nullptr) {
        int a = 0, b = 0;
        if (x->left != nullptr) {
          a = w(x->left);
        }
        if (x->right != nullptr) { 
          b = w(x->right);
        }
        return a + b + 1;
      }
    }
  \end{lstlisting}
\item sample text
  \begin{lstlisting}[language=c++]
    int kth (Node* x, int k) {
      Node* node = x;
      while (node != nullptr) {
        int a = 0, b = 0;
        if (x->left != nullptr) {
          a = w(x->left);
        }
        if (x->right != nullptr) { 
          b = w(x->right);
        }
        if (k == a + 1) {
          return node->key;
        } else if (k < a) {
          node = node->left;
        } else {
          node = node->right;
          k -= b + 1;
        }
      }
    }
  \end{lstlisting}
\item Найдем банальной операцией ноду с ключом x, результат - \[w(search(x).left)\]
  Или: 
  \begin{lstlisting}[language=c++]
    int result = 0;
    int func(Node* node, int x) {
      if (node == nullptr) {
        return;
      }

      if (node->value >= x) {
        func(node->left, x);
      } else {
        result += w(node);
        func(node->right, x);
      }
    }
  \end{lstlisting}
\item $[8, 3, 4]$ ?? если правильно понял данное условие 
  \item Да. Доказывается так же, как и высота АВЛ дерева. 
\end{enumerate}
\end{document}
