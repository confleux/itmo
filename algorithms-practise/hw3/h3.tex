\documentclass{article}

\usepackage[utf8]{inputenc}
\usepackage[T2A]{fontenc}
\usepackage[russian,english]{babel}
\usepackage{fontspec}
\usepackage{amsfonts}

\usepackage{enumitem}
\usepackage{amsmath}
\usepackage{amsthm}
\usepackage{listings}
\usepackage[left=2cm,right=2cm,bottom=2cm,top=2cm]{geometry}

\usepackage{algorithm}
\usepackage{algpseudocode}

\setmainfont{PT Serif}

\title{ДЗ по алгоритмам № 3. Кривенко Андрей М3107}

\begin{document}
\maketitle
\begin{enumerate}  
  \item \begin{lstlisting}[language=c++]
    int binSearchModified (int* arr, int l, int r, int x) {
      int mid = (l + r) / 2;
    
      if (l == r) {
        return mid;
      }
    
      if (arr[mid] == x) {
        return mid + 1;
      } else if (arr[mid] < x) {
        return binSearchModified(arr, mid + 1, r, x);
      } else {
        return binSearchModified(arr, l, mid, x);
      }
    }
    
    int algo (int* arr, int n, int x) {
      int* tmp = new int[n];
      tmp[0] = arr[0];
      for (int i = 1; i < n; ++i) {
        tmp[i] = tmp[i - 1] + arr[i];
      }
    
      return binSearchModified(tmp, 0, n, x);
    }
    \end{lstlisting}
  \item Требуется найти разделяющий два массива элемент такой, что $a_{i - 1} < a_{i} \wedge  a_{i + 1} < a_{i}$. 
  После выполняем сортировку в 2 двух массивах по разным правилам: для возрастания и для убывания.
  
  \begin{lstlisting}[language=c++]
    int findSeparator (int* arr, int l, int r) {
      int mid = (l + r) / 2;
      if (l >= r) {
        return -1;
      }

      if (arr[mid - 1] < arr[mid] && arr[mid + 1] < arr[mid]){
        return mid;
      }

      if (arr[mid - 1]  < arr[mid]) {
        return findSeparator(arr, mid, r);
      } else {
        return findSeparator(arr, l, mid);
      }
    }

    int algo (int* arr, int n, int x) {
      int sep = findSeparator(arr, 0, n);

      if (x == arr[sep]) {
        return sep;
      }

      if (x <= arr[sep - 1]) {
      int l = 0, r = sep;
      int mid = (l + r) / 2;
      while (l < r - 1) {
        mid = (l + r) / 2;
        if (arr[mid] == x) {
          return mid;
        } else if (arr[mid] < x) {
          l = mid;
        } else {
          r = mid;
        }
      }
      }

      if (x <= arr[sep + 1]) {
        int l = sep, r = n;
        int mid = (l + r) / 2;
        while (l < r - 1) {
          mid = (l + r) / 2;
          if (arr[mid] == x) {
            return mid;
          } else if (arr[mid] > x) {
            l = mid;
          } else {
            r = mid;
          }
        }
      }
      return -1;
    }
  \end{lstlisting}
  \item Нет, так как при циклическом сдвиге последнее число, т.е. наибольшее в массиве, станет первым и массив будет таким, что:
  на первой позиции наибольший элемент, а далее элементы расположены по возрастанию $\Longrightarrow$ нельзя применить бинарный поиск
  \item Можно записывать в стек структуру с элементами: число, сумма чисел предыдущих элементов. Тогда для того, чтобы получить сумму чисел в стеке, необходмо будет вернуть сумму чисел из последнего элемента стэка.
  \item Проще написать псевдокод, чем реализацию на c++ из-за ньюансов.
    Для решения задачи 5 мы можем сначала перевести выражение из инфиксной нотации в постфиксную, и считать выражение уже в ней.
    Таким образом, решается сразу же и шестая задача. \\
    Алгоритм перевода из инфиксной нотации в постфиксную, на вход которого подается выражение (токеном назовем любой char из выражения): \\
    Предварительно создаем массив вывода и стэк операторов.

    \begin{algorithmic}
      \While{$Can \; read \; token$}
        \State $Read \; token \; A$
        \If{$A \; is \; number$}
          \State $Push \; A \; to \; output \; array$
        \EndIf 
        \If{$A \; is \; operator \; o_{1}$}
          \While{$There \; is \; operator \; o_{2} \; on \; top \; of \; operator \; stack \; \land o_{2} \; has \; greater \; precedence \; than \; o_{1}$}
            \State $Pop \; o_{2} \; from \; operator \; stack \; and \; put \; to \; output \; array$
          \EndWhile{}
          \State  $Push \; o_{1} \; to \; operator \; stack$
        \EndIf
        \If{$A \; is \; '('$}
          \State  $Push \; o_{1} \; to \; operator \; stack$
        \EndIf
        \If{$A \; is \; ')'$}
          \While{$Operator \; op_{1} \; at \; top \; of \; operator \; stack \; is \; not \; '('$}
            \State $Pop \; operator \; from \; operator \; stack \; to \; output \; array$
            \State $Pop \; '(' \; from \; operator \; stack $
          \EndWhile
          \While{$There \; are \; tokens \; on \; operator \; stack$}
            \State $Pop \; operator \; from  \; operator \; stack \; to \; output array$
          \EndWhile
        \EndIf
      \EndWhile{}
    \end{algorithmic}
    Алгоритм подсчета выражения в постфиксной нотации:
    \begin{algorithmic}
      \While{$Can \; read \; token \; from \; output \; array$}
        \State $Read \; token \; A$
        \If{$A \; is \; operand$}
          \State $Push \; A \; to \; stack$
        \EndIf
        \If{$A \; is \; operator$}
          \State $Pop \; stack \; and \; get \; X_{1}$
          \State $Pop \; stack \; and \; get \; X_{2}$
          \State $Count \; X = X_{1} \; A \; X_{2}, \; where \; A \; is \; operator$
          \State $Push \; X \; to \; stack $
        \EndIf
      \EndWhile
      \State //Remaining element of stack is final value
    \end{algorithmic}
    \item Решено в предыдущей задаче
    \item Для того, чтобы развернуть односвязный список, требуется в каждом эл-те списка указатель на следующий элемент заменить на указатель на предыдущий элемент.
    \item Пусть за одну итерацию первый указатель на ноду переходит к следующей ноде, а второй указатель не две ноды вперед. Тогда, если эти два указателя встретятся, то список имеет кольцевой цикл.
    \item Можем использовать алгоритм из предыдущей задачи. В списке будет должно быть два цикла в двух разных направлениях.
    \item Сравниваем первые элементы двух отсортированных связных списков. 
      Найти наименьшую голову среду первых эл-тов.
      Все же прочие эл-ты списков будут больше этой голову.
      Теперь запустим рекурсивную функцию, приминающую как параметры следующую ноду наименьшей головы и другую голову и возвращающую следующий наименьший эл-т связного списка.
      Таким образом: currentelement.next = recursivefunc()
\end{enumerate}
\end{document}
