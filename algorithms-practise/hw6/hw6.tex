\documentclass{article}

\usepackage[utf8]{inputenc}
\usepackage[T2A]{fontenc}
\usepackage[russian,english]{babel}
\usepackage{fontspec}
\usepackage{amsfonts}

\usepackage{enumitem}
\usepackage{amsmath}
\usepackage{amsthm}
\usepackage{listings}
\usepackage[left=2cm,right=2cm,bottom=2cm,top=2cm]{geometry}

\setmainfont{PT Serif}

\title{ДЗ 6 по алгоритмам. Кривенко Андрей. М3107}

\begin{document}
\maketitle
\begin{enumerate}
  \item Пусть для какой-то из функций это не выполняется, тогда в одну ячейку может попасть $n-k$ элементов, где k - количество элементов, попавших в другие ячейки. Тогда если $k=n-1$, то в ячейку попадет 1 элемент. Тогда мы можем сказать, что коллизий не происходит. Однако функций без коллизий не существует(кроме тех, чье множество значений функции имеет большую мощность, чем область определения).
  \item Вставка в отсортированный список происхходит за $O(n)$, а не $O(1)$ в случает неотсортированного варианта.
  \item Будем использовать связный список, храня значение предыдущей и следующей добавленной ноды. Эта задача была в лабе на тему хэшей.
  \item При добавлении элемента в очередь будем высчитывать хэш, а затем в хэш таблицу записывать значение. Для разрешения коллизий можно использовать связный список. При добавлении, проверяем наличие элемента в таблице. Если его нет, то добавляем его и прибавляем к внешнему счетчику 1.При удалении из очереди точно так же проверять и если таких элементов больше нет, вычитать из счетчика 1.
  \item Пусть у нас есть таблица с n ключей и m ячеек. Тогда вероятность того, что у хэши двух ключей совпадают равна $\frac{1}{m}$. Будем использовать $I_{ij}$, равное 1, если ключи i и j имеют одинаковый хэш. Тогда матожидание $E[I_{ij}]$ равно $\frac{1}{m}$. Количество способов выбрать два элемента из n равно $\frac{n(n-1)}{2}$, тогда математическое ожидание равно: 
    \[ E[\Sigma_{i=j} I_{ij}] = \Sigma_{i=j} E[I_{ij}] = \Sigma_{i=j} I_{ij} = \Sigma_{ij} \frac{1}{m} = \frac{n(n-1)}{2m}\]
  \item В случае удаления, например, первого элемента, то после, при поиске второго, третьего и т.д. эл-та с таким же хешем, они не будут найдены, посколльку без флага мы не знаем о существовании других элементов с таким же хешем.
  \item Вычисляем хеши с помощью функций $h1(x)$ и  $h2(x)$. Если какая-либо из этих ячеек с индексами равными хешам свободна, то туда помещается элемент. В противном случае случайным образом выберем другую ячейку. Запомним ее значение, поместим туда наше изначальное. Теперь той же процедурой вставляем запомненный элемент. Если не вставили, то надо проверить то, зациклились ли мы. Если зациклились, то выберем новые 2 функции и перехешируем все эл-ты.  Когда таблица полностью заполнена, добавим новые ячейки.
\end{enumerate}
\end{document}
