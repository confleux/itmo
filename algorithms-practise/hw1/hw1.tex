\documentclass{article}

\usepackage[utf8]{inputenc}
\usepackage[T2A]{fontenc}
\usepackage[russian,english]{babel}
\usepackage{fontspec}
\usepackage{amsfonts}

\usepackage{enumitem}
\usepackage{amsmath}
\usepackage{amsthm}
\usepackage{listings}

\setmainfont{PT Serif}

\title{ДЗ по алгоритмам. Кривенко Андрей М3107}

\begin{document}
\maketitle
\begin{enumerate}  
  \item \begin{lstlisting}[language=c++]
for (int i = 0; i < 2n; ++i)
  for (int j = 0; j < n; ++j)
    for (int k = 0; k * k < 5n; ++k) 
    \end{lstlisting}
  \item , 
    \begin{lstlisting}[language=python]
i=0
while i < n:          //n + 1
  if i < n / 2:       //n
    j = 0             //n / 2
    while j * j < i:  //sum(from i = 0 to n / 2 of sqrt(i)) 
      j++             //sum(from i = 0 to n / 2 of sqrt(i)) 
  else: 
    j = i / 3         //n / 2
    while j / 2 < i:  //17(n / 2), j / 2 < i <=> j < 2i => требуется j += i / 10 20 раз, но j изначально j = i / 3 
      j += i / 10     //17(n / 2)
  i += 1              //n
    \end{lstlisting}
    \[ T(n) = (n + 1) + n + (n / 2) + \sum_{i=0}^{n / 2} \sqrt i +  \sum_{i=0}^{n / 2} \sqrt i + (n / 2) + 17(n / 2) + 17(n / 2) + n = \]
    \[ 21n + 2 \sum_{i=0}^{n / 2} \sqrt i + 1\]
    $O(n) = n \sqrt n$, так как внешний while запускается n раз, а вложенный while ~  n раз
  \item Докажите, что $\forall x > 0 : log_x n = \theta(\ln n)$
    \begin{proof}
      \[ \forall x > 0 : log_x n = \theta(\ln n) \iff \exists  c_1, c_2, n_0:\]
      $\begin{cases}
        c_1 \ln n \leq \log_x n \leq c_2 \ln n \\
        n \geq n_0 \\
        x > 0, x \neq 1
       \end{cases}$
       \begin{enumerate}
         \item При $n = 1: c_1 \ln n = \log_x n = c_2 \ln n = 0$
         \item Логарифмическая функция монотонно растет
         \item $(c_1 \ln n)' = \frac{c_1}{n} , \; (c_2 \ln n)' = \frac{c_2}{n} , \;  (\log_x n)' = \frac{1}{n \ln x}$
         \\ Следовательно, мы можем выбирать $c_1,\; c_2$ так, чтобы скорость роста $c_1 \ln n$ была ниже скорости роста $\log_x n$, а скорость роста $c_2 \ln n$ была выше скорости роста $\log_x n$: \\
         При $n_0 = 1$ и $n \in \mathbb{N} \wedge n >= n_0$: $\frac{c_1}{n} \leq \frac{1}{n\ln x} \leq \frac{c_2}{n} \iff c_1 \leq \frac{1}{\ln x} \leq c_2$ \\
         $\forall a \in \mathbb{R} \; \exists \; b, c \in \mathbb{R} : a < b, a > c \Longrightarrow \exists \; c_1 , c_ 2 \in \mathbb{R} : c_1 \leq \frac{1}{\ln x} \leq c_2$
       \end{enumerate}
      
    \end{proof}
  \item Псевдокод алгоритма
    \begin{lstlisting}[language=c++]
int n, m;
int arr1[n];
int arr2[m];
    
int i = 0, j = 0;

bool isFound = false;

while (i < n && j < m) {
  if (arr1[i] < arr2[j]) {
    ++i;
  } else if (arr1[i] > arr2[j]) {
    ++j;
  } else {
    isFound = true;
    break;
  }
}
    \end{lstlisting}
  \item Инверсии 
    \begin{lstlisting}[language=c++, numbers=left, xleftmargin=3.5ex]
for (int i = 1; i < n; ++i) {      
  int key = a[i];
  int j = i - 1;
  while (j > -1  && key < a[j]) {   
    a[j + 1] = a[j];
    --j; 
  }
  a[j + 1] = key;
}
    \end{lstlisting}

    for работает за n (строка 1). \\ 
    Условие цикла while выполняется только тогда, когда элемент с большим индексом меньше элемента с меньшим индексом ($i < j \wedge a[j] < a[i]$). То есть, оно выполняется столько раз, сколько существует пар $(i, j): i < j \wedge a_i > a_j$, то есть k раз.
    Следовательно, сортировка вставками работает за O(n + k)

    \item
    \begin{enumerate}

      \item Сортировка вставками устойчива, так как при ее использовании не изменяется относительный порядок элементов. Это происходит из-за того, что элементы добавляются в отсортированную часть массива последовательно справа налево (при сортировке в порядке неубывания), а также из-за того, что сравнение нового элемента и элемента отсортированной части массива строгое.
      \item Сортировка пузырьком будет неустойчивой, если использовать нестрогое сравнение при сравнении элементов:
      \begin{lstlisting}[language=c++, numbers=left, xleftmargin=3.5ex]
for (int i = 0; i < n - 1; ++i) {
  for (int j = 0; j < n - i - 1; ++j) {
    if (a[j] >= a[j + 1]) {
      swap(a[j], a[j + 1]);
    }
  }
}
      \end{lstlisting}
    \end{enumerate}

  \item $T(n) = \lfloor \log_3 n \rfloor + 2$, так как функция запускается столько раз, сколько n можно поделить на 3, а так же когда функция принимает n = 1 и n = 0 \\ 
    При n = 0 $T(n) = 1$ \\
    $\theta (n) = \log_3 n$, так как $ \exists c_1, c_2 : c_1 \log_3 n \leq \log_3 n \leq c_2 \log_3 n$ 
\end{enumerate}
\end{document}
